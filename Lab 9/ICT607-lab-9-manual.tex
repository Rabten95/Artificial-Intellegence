\documentclass[11pt]{article}

    \usepackage[breakable]{tcolorbox}
    \usepackage{parskip} % Stop auto-indenting (to mimic markdown behaviour)
    

    % Basic figure setup, for now with no caption control since it's done
    % automatically by Pandoc (which extracts ![](path) syntax from Markdown).
    \usepackage{graphicx}
    % Maintain compatibility with old templates. Remove in nbconvert 6.0
    \let\Oldincludegraphics\includegraphics
    % Ensure that by default, figures have no caption (until we provide a
    % proper Figure object with a Caption API and a way to capture that
    % in the conversion process - todo).
    \usepackage{caption}
    \DeclareCaptionFormat{nocaption}{}
    \captionsetup{format=nocaption,aboveskip=0pt,belowskip=0pt}

    \usepackage{float}
    \floatplacement{figure}{H} % forces figures to be placed at the correct location
    \usepackage{xcolor} % Allow colors to be defined
    \usepackage{enumerate} % Needed for markdown enumerations to work
    \usepackage{geometry} % Used to adjust the document margins
    \usepackage{amsmath} % Equations
    \usepackage{amssymb} % Equations
    \usepackage{textcomp} % defines textquotesingle
    % Hack from http://tex.stackexchange.com/a/47451/13684:
    \AtBeginDocument{%
        \def\PYZsq{\textquotesingle}% Upright quotes in Pygmentized code
    }
    \usepackage{upquote} % Upright quotes for verbatim code
    \usepackage{eurosym} % defines \euro

    \usepackage{iftex}
    \ifPDFTeX
        \usepackage[T1]{fontenc}
        \IfFileExists{alphabeta.sty}{
              \usepackage{alphabeta}
          }{
              \usepackage[mathletters]{ucs}
              \usepackage[utf8x]{inputenc}
          }
    \else
        \usepackage{fontspec}
        \usepackage{unicode-math}
    \fi

    \usepackage{fancyvrb} % verbatim replacement that allows latex
    \usepackage{grffile} % extends the file name processing of package graphics
                         % to support a larger range
    \makeatletter % fix for old versions of grffile with XeLaTeX
    \@ifpackagelater{grffile}{2019/11/01}
    {
      % Do nothing on new versions
    }
    {
      \def\Gread@@xetex#1{%
        \IfFileExists{"\Gin@base".bb}%
        {\Gread@eps{\Gin@base.bb}}%
        {\Gread@@xetex@aux#1}%
      }
    }
    \makeatother
    \usepackage[Export]{adjustbox} % Used to constrain images to a maximum size
    \adjustboxset{max size={0.9\linewidth}{0.9\paperheight}}

    % The hyperref package gives us a pdf with properly built
    % internal navigation ('pdf bookmarks' for the table of contents,
    % internal cross-reference links, web links for URLs, etc.)
    \usepackage{hyperref}
    % The default LaTeX title has an obnoxious amount of whitespace. By default,
    % titling removes some of it. It also provides customization options.
    \usepackage{titling}
    \usepackage{longtable} % longtable support required by pandoc >1.10
    \usepackage{booktabs}  % table support for pandoc > 1.12.2
    \usepackage{array}     % table support for pandoc >= 2.11.3
    \usepackage{calc}      % table minipage width calculation for pandoc >= 2.11.1
    \usepackage[inline]{enumitem} % IRkernel/repr support (it uses the enumerate* environment)
    \usepackage[normalem]{ulem} % ulem is needed to support strikethroughs (\sout)
                                % normalem makes italics be italics, not underlines
    \usepackage{mathrsfs}
    

    
    % Colors for the hyperref package
    \definecolor{urlcolor}{rgb}{0,.145,.698}
    \definecolor{linkcolor}{rgb}{.71,0.21,0.01}
    \definecolor{citecolor}{rgb}{.12,.54,.11}

    % ANSI colors
    \definecolor{ansi-black}{HTML}{3E424D}
    \definecolor{ansi-black-intense}{HTML}{282C36}
    \definecolor{ansi-red}{HTML}{E75C58}
    \definecolor{ansi-red-intense}{HTML}{B22B31}
    \definecolor{ansi-green}{HTML}{00A250}
    \definecolor{ansi-green-intense}{HTML}{007427}
    \definecolor{ansi-yellow}{HTML}{DDB62B}
    \definecolor{ansi-yellow-intense}{HTML}{B27D12}
    \definecolor{ansi-blue}{HTML}{208FFB}
    \definecolor{ansi-blue-intense}{HTML}{0065CA}
    \definecolor{ansi-magenta}{HTML}{D160C4}
    \definecolor{ansi-magenta-intense}{HTML}{A03196}
    \definecolor{ansi-cyan}{HTML}{60C6C8}
    \definecolor{ansi-cyan-intense}{HTML}{258F8F}
    \definecolor{ansi-white}{HTML}{C5C1B4}
    \definecolor{ansi-white-intense}{HTML}{A1A6B2}
    \definecolor{ansi-default-inverse-fg}{HTML}{FFFFFF}
    \definecolor{ansi-default-inverse-bg}{HTML}{000000}

    % common color for the border for error outputs.
    \definecolor{outerrorbackground}{HTML}{FFDFDF}

    % commands and environments needed by pandoc snippets
    % extracted from the output of `pandoc -s`
    \providecommand{\tightlist}{%
      \setlength{\itemsep}{0pt}\setlength{\parskip}{0pt}}
    \DefineVerbatimEnvironment{Highlighting}{Verbatim}{commandchars=\\\{\}}
    % Add ',fontsize=\small' for more characters per line
    \newenvironment{Shaded}{}{}
    \newcommand{\KeywordTok}[1]{\textcolor[rgb]{0.00,0.44,0.13}{\textbf{{#1}}}}
    \newcommand{\DataTypeTok}[1]{\textcolor[rgb]{0.56,0.13,0.00}{{#1}}}
    \newcommand{\DecValTok}[1]{\textcolor[rgb]{0.25,0.63,0.44}{{#1}}}
    \newcommand{\BaseNTok}[1]{\textcolor[rgb]{0.25,0.63,0.44}{{#1}}}
    \newcommand{\FloatTok}[1]{\textcolor[rgb]{0.25,0.63,0.44}{{#1}}}
    \newcommand{\CharTok}[1]{\textcolor[rgb]{0.25,0.44,0.63}{{#1}}}
    \newcommand{\StringTok}[1]{\textcolor[rgb]{0.25,0.44,0.63}{{#1}}}
    \newcommand{\CommentTok}[1]{\textcolor[rgb]{0.38,0.63,0.69}{\textit{{#1}}}}
    \newcommand{\OtherTok}[1]{\textcolor[rgb]{0.00,0.44,0.13}{{#1}}}
    \newcommand{\AlertTok}[1]{\textcolor[rgb]{1.00,0.00,0.00}{\textbf{{#1}}}}
    \newcommand{\FunctionTok}[1]{\textcolor[rgb]{0.02,0.16,0.49}{{#1}}}
    \newcommand{\RegionMarkerTok}[1]{{#1}}
    \newcommand{\ErrorTok}[1]{\textcolor[rgb]{1.00,0.00,0.00}{\textbf{{#1}}}}
    \newcommand{\NormalTok}[1]{{#1}}

    % Additional commands for more recent versions of Pandoc
    \newcommand{\ConstantTok}[1]{\textcolor[rgb]{0.53,0.00,0.00}{{#1}}}
    \newcommand{\SpecialCharTok}[1]{\textcolor[rgb]{0.25,0.44,0.63}{{#1}}}
    \newcommand{\VerbatimStringTok}[1]{\textcolor[rgb]{0.25,0.44,0.63}{{#1}}}
    \newcommand{\SpecialStringTok}[1]{\textcolor[rgb]{0.73,0.40,0.53}{{#1}}}
    \newcommand{\ImportTok}[1]{{#1}}
    \newcommand{\DocumentationTok}[1]{\textcolor[rgb]{0.73,0.13,0.13}{\textit{{#1}}}}
    \newcommand{\AnnotationTok}[1]{\textcolor[rgb]{0.38,0.63,0.69}{\textbf{\textit{{#1}}}}}
    \newcommand{\CommentVarTok}[1]{\textcolor[rgb]{0.38,0.63,0.69}{\textbf{\textit{{#1}}}}}
    \newcommand{\VariableTok}[1]{\textcolor[rgb]{0.10,0.09,0.49}{{#1}}}
    \newcommand{\ControlFlowTok}[1]{\textcolor[rgb]{0.00,0.44,0.13}{\textbf{{#1}}}}
    \newcommand{\OperatorTok}[1]{\textcolor[rgb]{0.40,0.40,0.40}{{#1}}}
    \newcommand{\BuiltInTok}[1]{{#1}}
    \newcommand{\ExtensionTok}[1]{{#1}}
    \newcommand{\PreprocessorTok}[1]{\textcolor[rgb]{0.74,0.48,0.00}{{#1}}}
    \newcommand{\AttributeTok}[1]{\textcolor[rgb]{0.49,0.56,0.16}{{#1}}}
    \newcommand{\InformationTok}[1]{\textcolor[rgb]{0.38,0.63,0.69}{\textbf{\textit{{#1}}}}}
    \newcommand{\WarningTok}[1]{\textcolor[rgb]{0.38,0.63,0.69}{\textbf{\textit{{#1}}}}}


    % Define a nice break command that doesn't care if a line doesn't already
    % exist.
    \def\br{\hspace*{\fill} \\* }
    % Math Jax compatibility definitions
    \def\gt{>}
    \def\lt{<}
    \let\Oldtex\TeX
    \let\Oldlatex\LaTeX
    \renewcommand{\TeX}{\textrm{\Oldtex}}
    \renewcommand{\LaTeX}{\textrm{\Oldlatex}}
    % Document parameters
    % Document title
    \title{ICT607: Artificial Intelligence for Cybersecurity}
    \date{Experiment 9}
       
    
    
    
% Pygments definitions
\makeatletter
\def\PY@reset{\let\PY@it=\relax \let\PY@bf=\relax%
    \let\PY@ul=\relax \let\PY@tc=\relax%
    \let\PY@bc=\relax \let\PY@ff=\relax}
\def\PY@tok#1{\csname PY@tok@#1\endcsname}
\def\PY@toks#1+{\ifx\relax#1\empty\else%
    \PY@tok{#1}\expandafter\PY@toks\fi}
\def\PY@do#1{\PY@bc{\PY@tc{\PY@ul{%
    \PY@it{\PY@bf{\PY@ff{#1}}}}}}}
\def\PY#1#2{\PY@reset\PY@toks#1+\relax+\PY@do{#2}}

\@namedef{PY@tok@w}{\def\PY@tc##1{\textcolor[rgb]{0.73,0.73,0.73}{##1}}}
\@namedef{PY@tok@c}{\let\PY@it=\textit\def\PY@tc##1{\textcolor[rgb]{0.24,0.48,0.48}{##1}}}
\@namedef{PY@tok@cp}{\def\PY@tc##1{\textcolor[rgb]{0.61,0.40,0.00}{##1}}}
\@namedef{PY@tok@k}{\let\PY@bf=\textbf\def\PY@tc##1{\textcolor[rgb]{0.00,0.50,0.00}{##1}}}
\@namedef{PY@tok@kp}{\def\PY@tc##1{\textcolor[rgb]{0.00,0.50,0.00}{##1}}}
\@namedef{PY@tok@kt}{\def\PY@tc##1{\textcolor[rgb]{0.69,0.00,0.25}{##1}}}
\@namedef{PY@tok@o}{\def\PY@tc##1{\textcolor[rgb]{0.40,0.40,0.40}{##1}}}
\@namedef{PY@tok@ow}{\let\PY@bf=\textbf\def\PY@tc##1{\textcolor[rgb]{0.67,0.13,1.00}{##1}}}
\@namedef{PY@tok@nb}{\def\PY@tc##1{\textcolor[rgb]{0.00,0.50,0.00}{##1}}}
\@namedef{PY@tok@nf}{\def\PY@tc##1{\textcolor[rgb]{0.00,0.00,1.00}{##1}}}
\@namedef{PY@tok@nc}{\let\PY@bf=\textbf\def\PY@tc##1{\textcolor[rgb]{0.00,0.00,1.00}{##1}}}
\@namedef{PY@tok@nn}{\let\PY@bf=\textbf\def\PY@tc##1{\textcolor[rgb]{0.00,0.00,1.00}{##1}}}
\@namedef{PY@tok@ne}{\let\PY@bf=\textbf\def\PY@tc##1{\textcolor[rgb]{0.80,0.25,0.22}{##1}}}
\@namedef{PY@tok@nv}{\def\PY@tc##1{\textcolor[rgb]{0.10,0.09,0.49}{##1}}}
\@namedef{PY@tok@no}{\def\PY@tc##1{\textcolor[rgb]{0.53,0.00,0.00}{##1}}}
\@namedef{PY@tok@nl}{\def\PY@tc##1{\textcolor[rgb]{0.46,0.46,0.00}{##1}}}
\@namedef{PY@tok@ni}{\let\PY@bf=\textbf\def\PY@tc##1{\textcolor[rgb]{0.44,0.44,0.44}{##1}}}
\@namedef{PY@tok@na}{\def\PY@tc##1{\textcolor[rgb]{0.41,0.47,0.13}{##1}}}
\@namedef{PY@tok@nt}{\let\PY@bf=\textbf\def\PY@tc##1{\textcolor[rgb]{0.00,0.50,0.00}{##1}}}
\@namedef{PY@tok@nd}{\def\PY@tc##1{\textcolor[rgb]{0.67,0.13,1.00}{##1}}}
\@namedef{PY@tok@s}{\def\PY@tc##1{\textcolor[rgb]{0.73,0.13,0.13}{##1}}}
\@namedef{PY@tok@sd}{\let\PY@it=\textit\def\PY@tc##1{\textcolor[rgb]{0.73,0.13,0.13}{##1}}}
\@namedef{PY@tok@si}{\let\PY@bf=\textbf\def\PY@tc##1{\textcolor[rgb]{0.64,0.35,0.47}{##1}}}
\@namedef{PY@tok@se}{\let\PY@bf=\textbf\def\PY@tc##1{\textcolor[rgb]{0.67,0.36,0.12}{##1}}}
\@namedef{PY@tok@sr}{\def\PY@tc##1{\textcolor[rgb]{0.64,0.35,0.47}{##1}}}
\@namedef{PY@tok@ss}{\def\PY@tc##1{\textcolor[rgb]{0.10,0.09,0.49}{##1}}}
\@namedef{PY@tok@sx}{\def\PY@tc##1{\textcolor[rgb]{0.00,0.50,0.00}{##1}}}
\@namedef{PY@tok@m}{\def\PY@tc##1{\textcolor[rgb]{0.40,0.40,0.40}{##1}}}
\@namedef{PY@tok@gh}{\let\PY@bf=\textbf\def\PY@tc##1{\textcolor[rgb]{0.00,0.00,0.50}{##1}}}
\@namedef{PY@tok@gu}{\let\PY@bf=\textbf\def\PY@tc##1{\textcolor[rgb]{0.50,0.00,0.50}{##1}}}
\@namedef{PY@tok@gd}{\def\PY@tc##1{\textcolor[rgb]{0.63,0.00,0.00}{##1}}}
\@namedef{PY@tok@gi}{\def\PY@tc##1{\textcolor[rgb]{0.00,0.52,0.00}{##1}}}
\@namedef{PY@tok@gr}{\def\PY@tc##1{\textcolor[rgb]{0.89,0.00,0.00}{##1}}}
\@namedef{PY@tok@ge}{\let\PY@it=\textit}
\@namedef{PY@tok@gs}{\let\PY@bf=\textbf}
\@namedef{PY@tok@gp}{\let\PY@bf=\textbf\def\PY@tc##1{\textcolor[rgb]{0.00,0.00,0.50}{##1}}}
\@namedef{PY@tok@go}{\def\PY@tc##1{\textcolor[rgb]{0.44,0.44,0.44}{##1}}}
\@namedef{PY@tok@gt}{\def\PY@tc##1{\textcolor[rgb]{0.00,0.27,0.87}{##1}}}
\@namedef{PY@tok@err}{\def\PY@bc##1{{\setlength{\fboxsep}{\string -\fboxrule}\fcolorbox[rgb]{1.00,0.00,0.00}{1,1,1}{\strut ##1}}}}
\@namedef{PY@tok@kc}{\let\PY@bf=\textbf\def\PY@tc##1{\textcolor[rgb]{0.00,0.50,0.00}{##1}}}
\@namedef{PY@tok@kd}{\let\PY@bf=\textbf\def\PY@tc##1{\textcolor[rgb]{0.00,0.50,0.00}{##1}}}
\@namedef{PY@tok@kn}{\let\PY@bf=\textbf\def\PY@tc##1{\textcolor[rgb]{0.00,0.50,0.00}{##1}}}
\@namedef{PY@tok@kr}{\let\PY@bf=\textbf\def\PY@tc##1{\textcolor[rgb]{0.00,0.50,0.00}{##1}}}
\@namedef{PY@tok@bp}{\def\PY@tc##1{\textcolor[rgb]{0.00,0.50,0.00}{##1}}}
\@namedef{PY@tok@fm}{\def\PY@tc##1{\textcolor[rgb]{0.00,0.00,1.00}{##1}}}
\@namedef{PY@tok@vc}{\def\PY@tc##1{\textcolor[rgb]{0.10,0.09,0.49}{##1}}}
\@namedef{PY@tok@vg}{\def\PY@tc##1{\textcolor[rgb]{0.10,0.09,0.49}{##1}}}
\@namedef{PY@tok@vi}{\def\PY@tc##1{\textcolor[rgb]{0.10,0.09,0.49}{##1}}}
\@namedef{PY@tok@vm}{\def\PY@tc##1{\textcolor[rgb]{0.10,0.09,0.49}{##1}}}
\@namedef{PY@tok@sa}{\def\PY@tc##1{\textcolor[rgb]{0.73,0.13,0.13}{##1}}}
\@namedef{PY@tok@sb}{\def\PY@tc##1{\textcolor[rgb]{0.73,0.13,0.13}{##1}}}
\@namedef{PY@tok@sc}{\def\PY@tc##1{\textcolor[rgb]{0.73,0.13,0.13}{##1}}}
\@namedef{PY@tok@dl}{\def\PY@tc##1{\textcolor[rgb]{0.73,0.13,0.13}{##1}}}
\@namedef{PY@tok@s2}{\def\PY@tc##1{\textcolor[rgb]{0.73,0.13,0.13}{##1}}}
\@namedef{PY@tok@sh}{\def\PY@tc##1{\textcolor[rgb]{0.73,0.13,0.13}{##1}}}
\@namedef{PY@tok@s1}{\def\PY@tc##1{\textcolor[rgb]{0.73,0.13,0.13}{##1}}}
\@namedef{PY@tok@mb}{\def\PY@tc##1{\textcolor[rgb]{0.40,0.40,0.40}{##1}}}
\@namedef{PY@tok@mf}{\def\PY@tc##1{\textcolor[rgb]{0.40,0.40,0.40}{##1}}}
\@namedef{PY@tok@mh}{\def\PY@tc##1{\textcolor[rgb]{0.40,0.40,0.40}{##1}}}
\@namedef{PY@tok@mi}{\def\PY@tc##1{\textcolor[rgb]{0.40,0.40,0.40}{##1}}}
\@namedef{PY@tok@il}{\def\PY@tc##1{\textcolor[rgb]{0.40,0.40,0.40}{##1}}}
\@namedef{PY@tok@mo}{\def\PY@tc##1{\textcolor[rgb]{0.40,0.40,0.40}{##1}}}
\@namedef{PY@tok@ch}{\let\PY@it=\textit\def\PY@tc##1{\textcolor[rgb]{0.24,0.48,0.48}{##1}}}
\@namedef{PY@tok@cm}{\let\PY@it=\textit\def\PY@tc##1{\textcolor[rgb]{0.24,0.48,0.48}{##1}}}
\@namedef{PY@tok@cpf}{\let\PY@it=\textit\def\PY@tc##1{\textcolor[rgb]{0.24,0.48,0.48}{##1}}}
\@namedef{PY@tok@c1}{\let\PY@it=\textit\def\PY@tc##1{\textcolor[rgb]{0.24,0.48,0.48}{##1}}}
\@namedef{PY@tok@cs}{\let\PY@it=\textit\def\PY@tc##1{\textcolor[rgb]{0.24,0.48,0.48}{##1}}}

\def\PYZbs{\char`\\}
\def\PYZus{\char`\_}
\def\PYZob{\char`\{}
\def\PYZcb{\char`\}}
\def\PYZca{\char`\^}
\def\PYZam{\char`\&}
\def\PYZlt{\char`\<}
\def\PYZgt{\char`\>}
\def\PYZsh{\char`\#}
\def\PYZpc{\char`\%}
\def\PYZdl{\char`\$}
\def\PYZhy{\char`\-}
\def\PYZsq{\char`\'}
\def\PYZdq{\char`\"}
\def\PYZti{\char`\~}
% for compatibility with earlier versions
\def\PYZat{@}
\def\PYZlb{[}
\def\PYZrb{]}
\makeatother


    % For linebreaks inside Verbatim environment from package fancyvrb.
    \makeatletter
        \newbox\Wrappedcontinuationbox
        \newbox\Wrappedvisiblespacebox
        \newcommand*\Wrappedvisiblespace {\textcolor{red}{\textvisiblespace}}
        \newcommand*\Wrappedcontinuationsymbol {\textcolor{red}{\llap{\tiny$\m@th\hookrightarrow$}}}
        \newcommand*\Wrappedcontinuationindent {3ex }
        \newcommand*\Wrappedafterbreak {\kern\Wrappedcontinuationindent\copy\Wrappedcontinuationbox}
        % Take advantage of the already applied Pygments mark-up to insert
        % potential linebreaks for TeX processing.
        %        {, <, #, %, $, ' and ": go to next line.
        %        _, }, ^, &, >, - and ~: stay at end of broken line.
        % Use of \textquotesingle for straight quote.
        \newcommand*\Wrappedbreaksatspecials {%
            \def\PYGZus{\discretionary{\char`\_}{\Wrappedafterbreak}{\char`\_}}%
            \def\PYGZob{\discretionary{}{\Wrappedafterbreak\char`\{}{\char`\{}}%
            \def\PYGZcb{\discretionary{\char`\}}{\Wrappedafterbreak}{\char`\}}}%
            \def\PYGZca{\discretionary{\char`\^}{\Wrappedafterbreak}{\char`\^}}%
            \def\PYGZam{\discretionary{\char`\&}{\Wrappedafterbreak}{\char`\&}}%
            \def\PYGZlt{\discretionary{}{\Wrappedafterbreak\char`\<}{\char`\<}}%
            \def\PYGZgt{\discretionary{\char`\>}{\Wrappedafterbreak}{\char`\>}}%
            \def\PYGZsh{\discretionary{}{\Wrappedafterbreak\char`\#}{\char`\#}}%
            \def\PYGZpc{\discretionary{}{\Wrappedafterbreak\char`\%}{\char`\%}}%
            \def\PYGZdl{\discretionary{}{\Wrappedafterbreak\char`\$}{\char`\$}}%
            \def\PYGZhy{\discretionary{\char`\-}{\Wrappedafterbreak}{\char`\-}}%
            \def\PYGZsq{\discretionary{}{\Wrappedafterbreak\textquotesingle}{\textquotesingle}}%
            \def\PYGZdq{\discretionary{}{\Wrappedafterbreak\char`\"}{\char`\"}}%
            \def\PYGZti{\discretionary{\char`\~}{\Wrappedafterbreak}{\char`\~}}%
        }
        % Some characters . , ; ? ! / are not pygmentized.
        % This macro makes them "active" and they will insert potential linebreaks
        \newcommand*\Wrappedbreaksatpunct {%
            \lccode`\~`\.\lowercase{\def~}{\discretionary{\hbox{\char`\.}}{\Wrappedafterbreak}{\hbox{\char`\.}}}%
            \lccode`\~`\,\lowercase{\def~}{\discretionary{\hbox{\char`\,}}{\Wrappedafterbreak}{\hbox{\char`\,}}}%
            \lccode`\~`\;\lowercase{\def~}{\discretionary{\hbox{\char`\;}}{\Wrappedafterbreak}{\hbox{\char`\;}}}%
            \lccode`\~`\:\lowercase{\def~}{\discretionary{\hbox{\char`\:}}{\Wrappedafterbreak}{\hbox{\char`\:}}}%
            \lccode`\~`\?\lowercase{\def~}{\discretionary{\hbox{\char`\?}}{\Wrappedafterbreak}{\hbox{\char`\?}}}%
            \lccode`\~`\!\lowercase{\def~}{\discretionary{\hbox{\char`\!}}{\Wrappedafterbreak}{\hbox{\char`\!}}}%
            \lccode`\~`\/\lowercase{\def~}{\discretionary{\hbox{\char`\/}}{\Wrappedafterbreak}{\hbox{\char`\/}}}%
            \catcode`\.\active
            \catcode`\,\active
            \catcode`\;\active
            \catcode`\:\active
            \catcode`\?\active
            \catcode`\!\active
            \catcode`\/\active
            \lccode`\~`\~
        }
    \makeatother

    \let\OriginalVerbatim=\Verbatim
    \makeatletter
    \renewcommand{\Verbatim}[1][1]{%
        %\parskip\z@skip
        \sbox\Wrappedcontinuationbox {\Wrappedcontinuationsymbol}%
        \sbox\Wrappedvisiblespacebox {\FV@SetupFont\Wrappedvisiblespace}%
        \def\FancyVerbFormatLine ##1{\hsize\linewidth
            \vtop{\raggedright\hyphenpenalty\z@\exhyphenpenalty\z@
                \doublehyphendemerits\z@\finalhyphendemerits\z@
                \strut ##1\strut}%
        }%
        % If the linebreak is at a space, the latter will be displayed as visible
        % space at end of first line, and a continuation symbol starts next line.
        % Stretch/shrink are however usually zero for typewriter font.
        \def\FV@Space {%
            \nobreak\hskip\z@ plus\fontdimen3\font minus\fontdimen4\font
            \discretionary{\copy\Wrappedvisiblespacebox}{\Wrappedafterbreak}
            {\kern\fontdimen2\font}%
        }%

        % Allow breaks at special characters using \PYG... macros.
        \Wrappedbreaksatspecials
        % Breaks at punctuation characters . , ; ? ! and / need catcode=\active
        \OriginalVerbatim[#1,codes*=\Wrappedbreaksatpunct]%
    }
    \makeatother

    % Exact colors from NB
    \definecolor{incolor}{HTML}{303F9F}
    \definecolor{outcolor}{HTML}{D84315}
    \definecolor{cellborder}{HTML}{CFCFCF}
    \definecolor{cellbackground}{HTML}{F7F7F7}

    % prompt
    \makeatletter
    \newcommand{\boxspacing}{\kern\kvtcb@left@rule\kern\kvtcb@boxsep}
    \makeatother
    \newcommand{\prompt}[4]{
        {\ttfamily\llap{{\color{#2}[#3]:\hspace{3pt}#4}}\vspace{-\baselineskip}}
    }
    

    
    % Prevent overflowing lines due to hard-to-break entities
    \sloppy
    % Setup hyperref package
    \hypersetup{
      breaklinks=true,  % so long urls are correctly broken across lines
      colorlinks=true,
      urlcolor=urlcolor,
      linkcolor=linkcolor,
      citecolor=citecolor,
      }
    % Slightly bigger margins than the latex defaults
    
    \geometry{verbose,tmargin=1in,bmargin=1in,lmargin=1in,rmargin=1in}
    
    

\begin{document}
    
    \maketitle
    
    

    
    

    \hypertarget{ai-driven-anomaly-detection-leveraging-forecasting-technique}{%
\section{AI-driven anomaly detection: leveraging forecasting
technique}\label{ai-driven-anomaly-detection-leveraging-forecasting-technique}}

Forecasting is a technique used to predict future trends based on
historical data. In the context of cybersecurity, forecasting can be
used to identify patterns and anomalies that may indicate a potential
security threat. By analyzing patterns in data over time, forecasting
algorithms can help detect anomalies in real-time, allowing
organizations to proactively respond to potential security incidents
before they become major issues.

In this lab, we will use Long Short-Term Memory (LSTM) forecasting
algorithm. LSTM is a type of recurrent neural network architecture that
is commonly used in deep learning applications. It is designed to handle
sequential data and is particularly well-suited for time-series
prediction tasks.

One of the main advantages of LSTM is its ability to remember
information for extended periods of time, which is important for tasks
that require long-term dependencies. This is achieved through the use of
gates that regulate the flow of information within the network.

    \textbf{Reference book:} Chio, C., \& Freeman, D. (2018). \emph{Machine
learning and security: Protecting systems with data and algorithms}
(First edition). O'Reilly Media. (page 95)

\textbf{Reference programs:}
https://github.com/oreilly-mlsec/book-resources/blob/master/chapter3/lstm-anomaly-detection.ipynb
(Modified and bug fixed)

    \begin{tcolorbox}[breakable, size=fbox, boxrule=1pt, pad at break*=1mm,colback=cellbackground, colframe=cellborder]
\prompt{In}{incolor}{1}{\boxspacing}
\begin{Verbatim}[commandchars=\\\{\}]
\PY{k+kn}{import} \PY{n+nn}{numpy} \PY{k}{as} \PY{n+nn}{np}
\PY{k+kn}{import} \PY{n+nn}{pandas} \PY{k}{as} \PY{n+nn}{pd}
\PY{k+kn}{from} \PY{n+nn}{datetime} \PY{k+kn}{import} \PY{n}{datetime}
\PY{k+kn}{import} \PY{n+nn}{matplotlib}\PY{n+nn}{.}\PY{n+nn}{pyplot} \PY{k}{as} \PY{n+nn}{plt}
\PY{k+kn}{import} \PY{n+nn}{time}
\end{Verbatim}
\end{tcolorbox}

    \begin{tcolorbox}[breakable, size=fbox, boxrule=1pt, pad at break*=1mm,colback=cellbackground, colframe=cellborder]
\prompt{In}{incolor}{2}{\boxspacing}
\begin{Verbatim}[commandchars=\\\{\}]
\PY{k+kn}{import} \PY{n+nn}{tensorflow} \PY{k}{as} \PY{n+nn}{tf}
\PY{k+kn}{from} \PY{n+nn}{tensorflow}\PY{n+nn}{.}\PY{n+nn}{keras}\PY{n+nn}{.}\PY{n+nn}{layers} \PY{k+kn}{import} \PY{n}{LSTM}\PY{p}{,} \PY{n}{Dropout}\PY{p}{,} \PY{n}{Dense}
\end{Verbatim}
\end{tcolorbox}

    \hypertarget{dataset}{%
\subsection{Dataset}\label{dataset}}

Download the dataset and keep it on your Google drive folder (e.g.,
\texttt{Colab\ Notebooks/AICS/} folder)

    \begin{tcolorbox}[breakable, size=fbox, boxrule=1pt, pad at break*=1mm,colback=cellbackground, colframe=cellborder]
\prompt{In}{incolor}{ }{\boxspacing}
\begin{Verbatim}[commandchars=\\\{\}]
\PY{k+kn}{from} \PY{n+nn}{google}\PY{n+nn}{.}\PY{n+nn}{colab} \PY{k+kn}{import} \PY{n}{drive}
\PY{n}{drive}\PY{o}{.}\PY{n}{mount}\PY{p}{(}\PY{l+s+s2}{\PYZdq{}}\PY{l+s+s2}{/content/drive}\PY{l+s+s2}{\PYZdq{}}\PY{p}{)}
\end{Verbatim}
\end{tcolorbox}

    \begin{tcolorbox}[breakable, size=fbox, boxrule=1pt, pad at break*=1mm,colback=cellbackground, colframe=cellborder]
\prompt{In}{incolor}{4}{\boxspacing}
\begin{Verbatim}[commandchars=\\\{\}]
\PY{n}{DATA} \PY{o}{=} \PY{n}{pd}\PY{o}{.}\PY{n}{read\PYZus{}csv}\PY{p}{(}
      \PY{l+s+s1}{\PYZsq{}}\PY{l+s+s1}{/content/drive/MyDrive/Colab Notebooks/AICS/cpu\PYZhy{}full\PYZhy{}b.csv}\PY{l+s+s1}{\PYZsq{}}\PY{p}{,}
      \PY{n}{parse\PYZus{}dates}\PY{o}{=}\PY{p}{[}\PY{l+m+mi}{0}\PY{p}{]}\PY{p}{,}
      \PY{n}{infer\PYZus{}datetime\PYZus{}format}\PY{o}{=}\PY{k+kc}{True}
  \PY{p}{)}
\end{Verbatim}
\end{tcolorbox}

    The second and third parameters \texttt{parse\_dates} and
\texttt{infer\_datetime\_format} are used to inform pandas how to handle
the date/time information in the CSV file.

The \texttt{parse\_dates} parameter in this case indicates that pandas
should parse the first column of the CSV file as dates.

The \texttt{infer\_datetime\_format} parameter is True, which means that
pandas will attempt to infer the date/time format from the CSV file.

    \begin{tcolorbox}[breakable, size=fbox, boxrule=1pt, pad at break*=1mm,colback=cellbackground, colframe=cellborder]
\prompt{In}{incolor}{5}{\boxspacing}
\begin{Verbatim}[commandchars=\\\{\}]
\PY{n}{DATA}\PY{o}{.}\PY{n}{head}\PY{p}{(}\PY{l+m+mi}{10}\PY{p}{)}
\end{Verbatim}
\end{tcolorbox}

            \begin{tcolorbox}[breakable, size=fbox, boxrule=.5pt, pad at break*=1mm, opacityfill=0]
\prompt{Out}{outcolor}{5}{\boxspacing}
\begin{Verbatim}[commandchars=\\\{\}]
             datetime   cpu
0 2017-01-27 18:42:00  1.14
1 2017-01-27 18:43:00  1.10
2 2017-01-27 18:44:00  1.09
3 2017-01-27 18:45:00  1.08
4 2017-01-27 18:46:00  1.08
5 2017-01-27 18:47:00  1.08
6 2017-01-27 18:48:00  1.15
7 2017-01-27 18:49:00  1.13
8 2017-01-27 18:50:00  1.09
9 2017-01-27 18:51:00  1.06
\end{Verbatim}
\end{tcolorbox}
        
    \begin{tcolorbox}[breakable, size=fbox, boxrule=1pt, pad at break*=1mm,colback=cellbackground, colframe=cellborder]
\prompt{In}{incolor}{6}{\boxspacing}
\begin{Verbatim}[commandchars=\\\{\}]
\PY{c+c1}{\PYZsh{} Each training data point will be length 100\PYZhy{}1,}
\PY{c+c1}{\PYZsh{} since the last value in each sequence is the label}
\PY{n}{SEQUENCE\PYZus{}LEN} \PY{o}{=} \PY{l+m+mi}{100}
\end{Verbatim}
\end{tcolorbox}

    \hypertarget{build-the-lstm-model}{%
\subsection{Build the LSTM model}\label{build-the-lstm-model}}

    \begin{tcolorbox}[breakable, size=fbox, boxrule=1pt, pad at break*=1mm,colback=cellbackground, colframe=cellborder]
\prompt{In}{incolor}{7}{\boxspacing}
\begin{Verbatim}[commandchars=\\\{\}]
\PY{k}{def} \PY{n+nf}{generate\PYZus{}model}\PY{p}{(}\PY{p}{)}\PY{p}{:}
  \PY{n}{model} \PY{o}{=} \PY{n}{tf}\PY{o}{.}\PY{n}{keras}\PY{o}{.}\PY{n}{Sequential}\PY{p}{(}\PY{p}{)}

  \PY{c+c1}{\PYZsh{} First LSTM layer defining the input sequence length}
  \PY{n}{model}\PY{o}{.}\PY{n}{add}\PY{p}{(}\PY{n}{LSTM}\PY{p}{(}
      \PY{n}{input\PYZus{}shape}\PY{o}{=}\PY{p}{(}\PY{n}{SEQUENCE\PYZus{}LEN}\PY{o}{\PYZhy{}}\PY{l+m+mi}{1}\PY{p}{,} \PY{l+m+mi}{1}\PY{p}{)}\PY{p}{,}
      \PY{n}{units}\PY{o}{=}\PY{l+m+mi}{32}\PY{p}{,}
      \PY{n}{return\PYZus{}sequences}\PY{o}{=}\PY{k+kc}{True}
  \PY{p}{)}\PY{p}{)}

  \PY{n}{model}\PY{o}{.}\PY{n}{add}\PY{p}{(}\PY{n}{Dropout}\PY{p}{(}\PY{l+m+mf}{0.2}\PY{p}{)}\PY{p}{)}

  \PY{c+c1}{\PYZsh{} Second LSTM layer with 128 units}
  \PY{n}{model}\PY{o}{.}\PY{n}{add}\PY{p}{(}\PY{n}{LSTM}\PY{p}{(}
      \PY{n}{units}\PY{o}{=}\PY{l+m+mi}{128}\PY{p}{,}
      \PY{n}{return\PYZus{}sequences}\PY{o}{=}\PY{k+kc}{True}
  \PY{p}{)}\PY{p}{)}

  \PY{n}{model}\PY{o}{.}\PY{n}{add}\PY{p}{(}\PY{n}{Dropout}\PY{p}{(}\PY{l+m+mf}{0.2}\PY{p}{)}\PY{p}{)}

  \PY{c+c1}{\PYZsh{} Third LSTM layer with 100 units}
  \PY{n}{model}\PY{o}{.}\PY{n}{add}\PY{p}{(}\PY{n}{LSTM}\PY{p}{(}
      \PY{n}{units}\PY{o}{=}\PY{l+m+mi}{100}\PY{p}{,}
      \PY{n}{return\PYZus{}sequences}\PY{o}{=}\PY{k+kc}{False}  \PY{c+c1}{\PYZsh{} no more sequence is required for the next layer}
  \PY{p}{)}\PY{p}{)}
  
  \PY{n}{model}\PY{o}{.}\PY{n}{add}\PY{p}{(}\PY{n}{Dropout}\PY{p}{(}\PY{l+m+mf}{0.2}\PY{p}{)}\PY{p}{)}

  \PY{c+c1}{\PYZsh{} Densely\PYZhy{}connected output layer with Linear activation function}
  \PY{n}{model}\PY{o}{.}\PY{n}{add}\PY{p}{(}\PY{n}{Dense}\PY{p}{(}
      \PY{n}{units}\PY{o}{=}\PY{l+m+mi}{1}\PY{p}{,}
      \PY{n}{activation}\PY{o}{=}\PY{l+s+s1}{\PYZsq{}}\PY{l+s+s1}{linear}\PY{l+s+s1}{\PYZsq{}}
  \PY{p}{)}\PY{p}{)}

  \PY{n}{model}\PY{o}{.}\PY{n}{compile}\PY{p}{(}
      \PY{n}{loss}\PY{o}{=}\PY{n}{tf}\PY{o}{.}\PY{n}{keras}\PY{o}{.}\PY{n}{losses}\PY{o}{.}\PY{n}{MeanSquaredError}\PY{p}{(}\PY{p}{)}\PY{p}{,}
      \PY{n}{optimizer}\PY{o}{=}\PY{n}{tf}\PY{o}{.}\PY{n}{keras}\PY{o}{.}\PY{n}{optimizers}\PY{o}{.}\PY{n}{RMSprop}\PY{p}{(}\PY{n}{learning\PYZus{}rate}\PY{o}{=}\PY{l+m+mf}{0.001}\PY{p}{)}
  \PY{p}{)}

  \PY{k}{return} \PY{n}{model}
\end{Verbatim}
\end{tcolorbox}

    The model consists of three LSTM layers with 32, 128, and 100 units
respectively, along with a densely-connected output layer with a linear
activation function. The input sequence length is defined by the
``SEQUENCE\_LEN'' variable minus one, and the model includes dropout
regularization with a rate of 0.2 to prevent overfitting.

The input shape is defined as (SEQUENCE\_LEN-1, 1) in the first LSTM
layer. The first dimension of the input shape (SEQUENCE\_LEN-1) refers
to the length of the input sequence, while the second dimension (1)
refers to the number of features in each time step of the input
sequence.

The model is compiled using the mean squared error loss function and the
RMSprop optimizer with a learning rate of 0.001.

    \begin{tcolorbox}[breakable, size=fbox, boxrule=1pt, pad at break*=1mm,colback=cellbackground, colframe=cellborder]
\prompt{In}{incolor}{8}{\boxspacing}
\begin{Verbatim}[commandchars=\\\{\}]
\PY{k}{def} \PY{n+nf}{normalise}\PY{p}{(}\PY{n}{result}\PY{p}{)}\PY{p}{:}
  \PY{n}{result\PYZus{}mean} \PY{o}{=} \PY{n}{result}\PY{o}{.}\PY{n}{mean}\PY{p}{(}\PY{p}{)}
  \PY{n}{result\PYZus{}std} \PY{o}{=} \PY{n}{result}\PY{o}{.}\PY{n}{std}\PY{p}{(}\PY{p}{)}
  \PY{n}{result} \PY{o}{\PYZhy{}}\PY{o}{=} \PY{n}{result\PYZus{}mean}
  \PY{n}{result} \PY{o}{/}\PY{o}{=} \PY{n}{result\PYZus{}std}
  \PY{k}{return} \PY{n}{result}
\end{Verbatim}
\end{tcolorbox}

    The function normalises the data by subtracting the mean and dividing by
the standard deviation. This is a common technique used to scale and
center data around zero, which can help improve the performance of
machine learning models.

    \hypertarget{data-preparation}{%
\subsection{Data preparation}\label{data-preparation}}

    \begin{tcolorbox}[breakable, size=fbox, boxrule=1pt, pad at break*=1mm,colback=cellbackground, colframe=cellborder]
\prompt{In}{incolor}{9}{\boxspacing}
\begin{Verbatim}[commandchars=\\\{\}]
\PY{k}{def} \PY{n+nf}{prepare\PYZus{}data}\PY{p}{(}\PY{n}{data}\PY{p}{,} \PY{n}{train\PYZus{}start}\PY{p}{,} \PY{n}{train\PYZus{}end}\PY{p}{,} \PY{n}{test\PYZus{}start}\PY{p}{,} \PY{n}{test\PYZus{}end}\PY{p}{)}\PY{p}{:}
  \PY{n+nb}{print}\PY{p}{(}\PY{l+s+s2}{\PYZdq{}}\PY{l+s+se}{\PYZbs{}n}\PY{l+s+s2}{Length of Data}\PY{l+s+s2}{\PYZdq{}}\PY{p}{,} \PY{n+nb}{len}\PY{p}{(}\PY{n}{data}\PY{p}{)}\PY{p}{)}

  \PY{c+c1}{\PYZsh{} training data}
  \PY{n}{result} \PY{o}{=} \PY{p}{[}\PY{p}{]}
  \PY{k}{for} \PY{n}{index} \PY{o+ow}{in} \PY{n+nb}{range}\PY{p}{(}\PY{n}{train\PYZus{}start}\PY{p}{,} \PY{n}{train\PYZus{}end}\PY{o}{\PYZhy{}}\PY{n}{SEQUENCE\PYZus{}LEN}\PY{p}{)}\PY{p}{:}
    \PY{n}{result}\PY{o}{.}\PY{n}{append}\PY{p}{(}\PY{n}{data}\PY{p}{[}\PY{n}{index}\PY{p}{:} \PY{n}{index} \PY{o}{+} \PY{n}{SEQUENCE\PYZus{}LEN}\PY{p}{]}\PY{p}{)}
  \PY{n}{result} \PY{o}{=} \PY{n}{np}\PY{o}{.}\PY{n}{array}\PY{p}{(}\PY{n}{result}\PY{p}{)}
  \PY{n}{result} \PY{o}{=} \PY{n}{normalise}\PY{p}{(}\PY{n}{result}\PY{p}{)}

  \PY{n+nb}{print}\PY{p}{(}\PY{l+s+s2}{\PYZdq{}}\PY{l+s+se}{\PYZbs{}n}\PY{l+s+s2}{Training data shape: }\PY{l+s+s2}{\PYZdq{}}\PY{p}{,} \PY{n}{result}\PY{o}{.}\PY{n}{shape}\PY{p}{)}
  \PY{n+nb}{print}\PY{p}{(}\PY{l+s+s2}{\PYZdq{}}\PY{l+s+s2}{Training data:}\PY{l+s+se}{\PYZbs{}n}\PY{l+s+s2}{\PYZdq{}}\PY{p}{,} \PY{n}{result}\PY{p}{)}

  \PY{n}{np}\PY{o}{.}\PY{n}{random}\PY{o}{.}\PY{n}{shuffle}\PY{p}{(}\PY{n}{result}\PY{p}{)}
  
  \PY{n}{X\PYZus{}train} \PY{o}{=} \PY{n}{result}\PY{p}{[}\PY{p}{:}\PY{p}{,} \PY{p}{:}\PY{o}{\PYZhy{}}\PY{l+m+mi}{1}\PY{p}{]}
  \PY{n}{y\PYZus{}train} \PY{o}{=} \PY{n}{result}\PY{p}{[}\PY{p}{:}\PY{p}{,} \PY{o}{\PYZhy{}}\PY{l+m+mi}{1}\PY{p}{]}

  \PY{c+c1}{\PYZsh{} test data}
  \PY{n}{result} \PY{o}{=} \PY{p}{[}\PY{p}{]}
  \PY{k}{for} \PY{n}{index} \PY{o+ow}{in} \PY{n+nb}{range}\PY{p}{(}\PY{n}{test\PYZus{}start}\PY{p}{,} \PY{n}{test\PYZus{}end}\PY{o}{\PYZhy{}}\PY{n}{SEQUENCE\PYZus{}LEN}\PY{p}{)}\PY{p}{:}
    \PY{n}{result}\PY{o}{.}\PY{n}{append}\PY{p}{(}\PY{n}{data}\PY{p}{[}\PY{n}{index}\PY{p}{:} \PY{n}{index} \PY{o}{+} \PY{n}{SEQUENCE\PYZus{}LEN}\PY{p}{]}\PY{p}{)}
  \PY{n}{result} \PY{o}{=} \PY{n}{np}\PY{o}{.}\PY{n}{array}\PY{p}{(}\PY{n}{result}\PY{p}{)}
  \PY{n}{result} \PY{o}{=} \PY{n}{normalise}\PY{p}{(}\PY{n}{result}\PY{p}{)}

  \PY{n+nb}{print}\PY{p}{(}\PY{l+s+s2}{\PYZdq{}}\PY{l+s+se}{\PYZbs{}n}\PY{l+s+s2}{Test data shape: }\PY{l+s+s2}{\PYZdq{}}\PY{p}{,} \PY{n}{result}\PY{o}{.}\PY{n}{shape}\PY{p}{)}
  \PY{n+nb}{print}\PY{p}{(}\PY{l+s+s2}{\PYZdq{}}\PY{l+s+se}{\PYZbs{}n}\PY{l+s+s2}{Test data:}\PY{l+s+se}{\PYZbs{}n}\PY{l+s+s2}{\PYZdq{}}\PY{p}{,} \PY{n}{result}\PY{p}{)}

  \PY{n}{X\PYZus{}test} \PY{o}{=} \PY{n}{result}\PY{p}{[}\PY{p}{:}\PY{p}{,} \PY{p}{:}\PY{o}{\PYZhy{}}\PY{l+m+mi}{1}\PY{p}{]}
  \PY{n}{y\PYZus{}test} \PY{o}{=} \PY{n}{result}\PY{p}{[}\PY{p}{:}\PY{p}{,} \PY{o}{\PYZhy{}}\PY{l+m+mi}{1}\PY{p}{]}

  \PY{n+nb}{print}\PY{p}{(}\PY{l+s+s2}{\PYZdq{}}\PY{l+s+se}{\PYZbs{}n}\PY{l+s+s2}{Shape X\PYZus{}train: }\PY{l+s+s2}{\PYZdq{}}\PY{p}{,} \PY{n}{np}\PY{o}{.}\PY{n}{shape}\PY{p}{(}\PY{n}{X\PYZus{}train}\PY{p}{)}\PY{p}{)}
  \PY{n+nb}{print}\PY{p}{(}\PY{l+s+s2}{\PYZdq{}}\PY{l+s+s2}{Shape X\PYZus{}test: }\PY{l+s+s2}{\PYZdq{}}\PY{p}{,} \PY{n}{np}\PY{o}{.}\PY{n}{shape}\PY{p}{(}\PY{n}{X\PYZus{}test}\PY{p}{)}\PY{p}{)}

  \PY{n}{X\PYZus{}train} \PY{o}{=} \PY{n}{np}\PY{o}{.}\PY{n}{reshape}\PY{p}{(}\PY{n}{X\PYZus{}train}\PY{p}{,} \PY{p}{(}\PY{n}{X\PYZus{}train}\PY{o}{.}\PY{n}{shape}\PY{p}{[}\PY{l+m+mi}{0}\PY{p}{]}\PY{p}{,} \PY{n}{X\PYZus{}train}\PY{o}{.}\PY{n}{shape}\PY{p}{[}\PY{l+m+mi}{1}\PY{p}{]}\PY{p}{,} \PY{l+m+mi}{1}\PY{p}{)}\PY{p}{)}
  \PY{n}{X\PYZus{}test} \PY{o}{=} \PY{n}{np}\PY{o}{.}\PY{n}{reshape}\PY{p}{(}\PY{n}{X\PYZus{}test}\PY{p}{,} \PY{p}{(}\PY{n}{X\PYZus{}test}\PY{o}{.}\PY{n}{shape}\PY{p}{[}\PY{l+m+mi}{0}\PY{p}{]}\PY{p}{,} \PY{n}{X\PYZus{}test}\PY{o}{.}\PY{n}{shape}\PY{p}{[}\PY{l+m+mi}{1}\PY{p}{]}\PY{p}{,} \PY{l+m+mi}{1}\PY{p}{)}\PY{p}{)}

  \PY{k}{return} \PY{n}{X\PYZus{}train}\PY{p}{,} \PY{n}{y\PYZus{}train}\PY{p}{,} \PY{n}{X\PYZus{}test}\PY{p}{,} \PY{n}{y\PYZus{}test}
\end{Verbatim}
\end{tcolorbox}

    The function first prints the length of the input data. It creates two
sets of data: one for training and one for testing.

For the training data, the function creates a list result of sequences
with length SEQUENCE\_LEN by iterating over the input data from
train\_start to train\_end. The result list is then converted to a numpy
array and normalised using the normalise function. It then shuffles the
normalised training data and separates the input features X\_train (all
columns except the last) from the target values y\_train (the last
column).

For the test data, the function follows the same procedure as for the
training data, but uses the input data from test\_start to test\_end
instead.

Finally, the function reshapes the input features X\_train and X\_test
to have an additional dimension of size 1.

Reshaping is necessary here because the input data X\_train and X\_test
need to be formatted to fit the input shape of the LSTM layers in the
model. Specifically, the LSTM layers in the model expect input data in a
3D format of shape (batch\_size, sequence\_length, input\_dim), where:

\begin{itemize}
\tightlist
\item
  batch\_size is the number of samples in each batch of data
\item
  sequence\_length is the length of each input sequence
\item
  input\_dim is the number of features in the input data
\end{itemize}

In the prepare\_data function, input (X) and target (y) arrays have
shapes (num\_samples, sequence\_length-1) and (num\_samples,),
respectively.

To prepare the data for input into the LSTM layers, we need to reshape
the input arrays to add an extra dimension for the input\_dim. This is
achieved by calling the reshape() function on X\_train and X\_test and
specifying the new shape to be (num\_samples, sequence\_length-1, 1).
This creates a new 3D array with the required shape to fit the LSTM
layers in the model.

    \hypertarget{running-the-lstm-model-with-necessary-plotting}{%
\subsection{Running the LSTM model with necessary
plotting}\label{running-the-lstm-model-with-necessary-plotting}}

    \begin{tcolorbox}[breakable, size=fbox, boxrule=1pt, pad at break*=1mm,colback=cellbackground, colframe=cellborder]
\prompt{In}{incolor}{10}{\boxspacing}
\begin{Verbatim}[commandchars=\\\{\}]
\PY{k}{def} \PY{n+nf}{run}\PY{p}{(}\PY{n}{model}\PY{o}{=}\PY{k+kc}{None}\PY{p}{,} \PY{n}{data}\PY{o}{=}\PY{k+kc}{None}\PY{p}{,} \PY{n}{epochs}\PY{o}{=}\PY{l+m+mi}{3}\PY{p}{,} \PY{n}{batch\PYZus{}size}\PY{o}{=}\PY{l+m+mi}{50}\PY{p}{)}\PY{p}{:}
  \PY{n}{global\PYZus{}start\PYZus{}time} \PY{o}{=} \PY{n}{time}\PY{o}{.}\PY{n}{time}\PY{p}{(}\PY{p}{)}

  \PY{n}{data} \PY{o}{=} \PY{n}{data}\PY{p}{[}\PY{l+s+s1}{\PYZsq{}}\PY{l+s+s1}{cpu}\PY{l+s+s1}{\PYZsq{}}\PY{p}{]}\PY{o}{.}\PY{n}{to\PYZus{}numpy}\PY{p}{(}\PY{p}{)}

  \PY{c+c1}{\PYZsh{} train on first 700 samples and test on next 300 samples}
  \PY{n}{X\PYZus{}train}\PY{p}{,} \PY{n}{y\PYZus{}train}\PY{p}{,} \PY{n}{X\PYZus{}test}\PY{p}{,} \PY{n}{y\PYZus{}test} \PY{o}{=} \PY{n}{prepare\PYZus{}data}\PY{p}{(}
      \PY{n}{data}\PY{o}{=}\PY{n}{data}\PY{p}{,}
      \PY{n}{train\PYZus{}start}\PY{o}{=}\PY{l+m+mi}{0}\PY{p}{,}
      \PY{n}{train\PYZus{}end}\PY{o}{=}\PY{l+m+mi}{400}\PY{p}{,}
      \PY{n}{test\PYZus{}start}\PY{o}{=}\PY{l+m+mi}{400}\PY{p}{,}
      \PY{n}{test\PYZus{}end}\PY{o}{=}\PY{l+m+mi}{660}
  \PY{p}{)}

  \PY{k}{if} \PY{n}{model} \PY{o+ow}{is} \PY{k+kc}{None}\PY{p}{:}
    \PY{n}{model} \PY{o}{=} \PY{n}{generate\PYZus{}model}\PY{p}{(}\PY{p}{)}

  \PY{k}{try}\PY{p}{:}
    \PY{n+nb}{print}\PY{p}{(}\PY{l+s+s2}{\PYZdq{}}\PY{l+s+se}{\PYZbs{}n}\PY{l+s+s2}{Training...}\PY{l+s+s2}{\PYZdq{}}\PY{p}{)}
    \PY{n}{model}\PY{o}{.}\PY{n}{fit}\PY{p}{(}
        \PY{n}{x}\PY{o}{=}\PY{n}{X\PYZus{}train}\PY{p}{,}
        \PY{n}{y}\PY{o}{=}\PY{n}{y\PYZus{}train}\PY{p}{,}
        \PY{n}{batch\PYZus{}size}\PY{o}{=}\PY{n}{batch\PYZus{}size}\PY{p}{,}
        \PY{n}{epochs}\PY{o}{=}\PY{n}{epochs}\PY{p}{,}
        \PY{n}{validation\PYZus{}split}\PY{o}{=}\PY{l+m+mf}{0.05}
    \PY{p}{)}
    \PY{n+nb}{print}\PY{p}{(}\PY{l+s+s2}{\PYZdq{}}\PY{l+s+se}{\PYZbs{}n}\PY{l+s+s2}{Predicting...}\PY{l+s+s2}{\PYZdq{}}\PY{p}{)}
    \PY{n}{predicted} \PY{o}{=} \PY{n}{model}\PY{o}{.}\PY{n}{predict}\PY{p}{(}\PY{n}{X\PYZus{}test}\PY{p}{)}

    \PY{n}{predicted} \PY{o}{=} \PY{n}{np}\PY{o}{.}\PY{n}{reshape}\PY{p}{(}\PY{n}{predicted}\PY{p}{,} \PY{p}{(}\PY{n}{predicted}\PY{o}{.}\PY{n}{size}\PY{p}{,}\PY{p}{)}\PY{p}{)}

  \PY{k}{except} \PY{n+ne}{KeyboardInterrupt}\PY{p}{:}
    \PY{n+nb}{print}\PY{p}{(}\PY{l+s+s2}{\PYZdq{}}\PY{l+s+s2}{prediction exception}\PY{l+s+s2}{\PYZdq{}}\PY{p}{)}
    \PY{n+nb}{print}\PY{p}{(}\PY{l+s+s1}{\PYZsq{}}\PY{l+s+s1}{Training duration:}\PY{l+s+si}{\PYZob{}\PYZcb{}}\PY{l+s+s1}{\PYZsq{}}\PY{o}{.}\PY{n}{format}\PY{p}{(}\PY{n}{time}\PY{o}{.}\PY{n}{time}\PY{p}{(}\PY{p}{)}\PY{o}{\PYZhy{}}\PY{n}{global\PYZus{}start\PYZus{}time}\PY{p}{)}\PY{p}{)}
    \PY{k}{return} \PY{n}{model}\PY{p}{,} \PY{n}{y\PYZus{}test}\PY{p}{,} \PY{l+m+mi}{0}

  \PY{k}{try}\PY{p}{:}
    \PY{n}{plt}\PY{o}{.}\PY{n}{figure}\PY{p}{(}\PY{n}{figsize}\PY{o}{=}\PY{p}{(}\PY{l+m+mi}{20}\PY{p}{,}\PY{l+m+mi}{8}\PY{p}{)}\PY{p}{)}
    \PY{n}{plt}\PY{o}{.}\PY{n}{plot}\PY{p}{(}\PY{n}{y\PYZus{}test}\PY{p}{,} \PY{l+s+s1}{\PYZsq{}}\PY{l+s+s1}{b}\PY{l+s+s1}{\PYZsq{}}\PY{p}{,} \PY{n}{label}\PY{o}{=}\PY{l+s+s1}{\PYZsq{}}\PY{l+s+s1}{Observed}\PY{l+s+s1}{\PYZsq{}}\PY{p}{)}
    \PY{n}{plt}\PY{o}{.}\PY{n}{plot}\PY{p}{(}\PY{n}{predicted}\PY{p}{,} \PY{l+s+s1}{\PYZsq{}}\PY{l+s+s1}{g}\PY{l+s+s1}{\PYZsq{}}\PY{p}{,} \PY{n}{label}\PY{o}{=}\PY{l+s+s1}{\PYZsq{}}\PY{l+s+s1}{Predicted}\PY{l+s+s1}{\PYZsq{}}\PY{p}{)}
    \PY{n}{plt}\PY{o}{.}\PY{n}{plot}\PY{p}{(}\PY{p}{(}\PY{p}{(}\PY{n}{y\PYZus{}test}\PY{o}{\PYZhy{}}\PY{n}{predicted}\PY{p}{)}\PY{o}{*}\PY{o}{*}\PY{l+m+mi}{2}\PY{p}{)}\PY{p}{,} \PY{l+s+s1}{\PYZsq{}}\PY{l+s+s1}{r}\PY{l+s+s1}{\PYZsq{}}\PY{p}{,} \PY{n}{label}\PY{o}{=}\PY{l+s+s2}{\PYZdq{}}\PY{l+s+s2}{Root\PYZhy{}mean\PYZhy{}square deviation}\PY{l+s+s2}{\PYZdq{}}\PY{p}{)}
    \PY{n}{plt}\PY{o}{.}\PY{n}{legend}\PY{p}{(}\PY{p}{)}
    \PY{n}{plt}\PY{o}{.}\PY{n}{show}\PY{p}{(}\PY{p}{)}
  \PY{k}{except} \PY{n+ne}{Exception} \PY{k}{as} \PY{n}{e}\PY{p}{:}
    \PY{n+nb}{print}\PY{p}{(}\PY{l+s+s2}{\PYZdq{}}\PY{l+s+s2}{plotting exception}\PY{l+s+s2}{\PYZdq{}}\PY{p}{)}
    \PY{n+nb}{print}\PY{p}{(}\PY{n+nb}{str}\PY{p}{(}\PY{n}{e}\PY{p}{)}\PY{p}{)}
  
  \PY{n+nb}{print}\PY{p}{(}\PY{l+s+s1}{\PYZsq{}}\PY{l+s+s1}{Training duration: }\PY{l+s+si}{\PYZob{}\PYZcb{}}\PY{l+s+s1}{\PYZsq{}}\PY{o}{.}\PY{n}{format}\PY{p}{(}\PY{n}{time}\PY{o}{.}\PY{n}{time}\PY{p}{(}\PY{p}{)}\PY{o}{\PYZhy{}}\PY{n}{global\PYZus{}start\PYZus{}time}\PY{p}{)}\PY{p}{)}

  \PY{k}{return} \PY{n}{model}\PY{p}{,} \PY{n}{y\PYZus{}test}\PY{p}{,} \PY{n}{predicted}
\end{Verbatim}
\end{tcolorbox}

    By setting train\_start=0 and train\_end=400, the first 400 data points
are used for training. Similarly, by setting test\_start=400 and
test\_end=660, the next 260 data points are used for testing. Therefore,
the model is trained on the first 400 time steps of the dataset and
tested on the subsequent 260 time steps. This split ensures that the
model is evaluated on unseen data and not overfitting to the training
dataset.

The purpose of predicted = np.reshape(predicted, (predicted.size,)) is
to convert the predicted values from a two-dimensional numpy array of
shape (n,1) to a one-dimensional numpy array of shape (n,), where n is
the number of predicted values.

In other words, the np.reshape() function is used to ``flatten'' the
predicted values array, which makes it easier to perform calculations on
the predicted values or to plot them against the actual values.

The first plot command plots the actual values of the test dataset as a
blue line with the label ``Observed''.

The second plot command plots the predicted values of the test dataset
as a green line with the label ``Predicted''.

The third plot command plots the root-mean-square deviation (RMSE)
between the predicted and actual values as a red line with the label
``Root-mean-square deviation''. The RMSE is calculated as the square
root of the mean squared error between the predicted and actual values.
((y\_test-predicted)**2) is used to calculate the squared error between
the predicted and actual values.

    \begin{tcolorbox}[breakable, size=fbox, boxrule=1pt, pad at break*=1mm,colback=cellbackground, colframe=cellborder]
\prompt{In}{incolor}{11}{\boxspacing}
\begin{Verbatim}[commandchars=\\\{\}]
\PY{n}{model}\PY{p}{,} \PY{n}{y\PYZus{}test}\PY{p}{,} \PY{n}{predicted} \PY{o}{=} \PY{n}{run}\PY{p}{(}\PY{n}{data}\PY{o}{=}\PY{n}{DATA}\PY{p}{)}
\end{Verbatim}
\end{tcolorbox}

    \begin{Verbatim}[commandchars=\\\{\}]

Length of Data 660

Training data shape:  (300, 100)
Training data:
 [[-0.08360845 -0.17863064 -0.20238619 {\ldots}  2.22067983  2.43447977
   2.60076861]
 [-0.17863064 -0.20238619 -0.22614174 {\ldots}  2.43447977  2.60076861
   1.79307994]
 [-0.20238619 -0.22614174 -0.22614174 {\ldots}  2.60076861  1.79307994
   1.31796896]
 {\ldots}
 [ 0.55779138  1.17543566  0.72408023 {\ldots}  2.38696867  1.88810214
   2.19692428]
 [ 1.17543566  0.72408023  0.41525809 {\ldots}  1.88810214  2.19692428
   3.21841289]
 [ 0.72408023  0.41525809  0.22521369 {\ldots}  2.19692428  3.21841289
   3.26592399]]

Test data shape:  (160, 100)

Test data:
 [[ 1.65059432  1.76888841  1.11827094 {\ldots} -0.85329712 -0.83358144
  -0.675856  ]
 [ 1.76888841  1.11827094  0.80282005 {\ldots} -0.83358144 -0.675856
  -0.675856  ]
 [ 1.11827094  0.80282005  1.39429047 {\ldots} -0.675856   -0.675856
  -0.65614032]
 {\ldots}
 [ 0.94082982  0.78310437  0.50708484 {\ldots}  2.24206474  2.34064315
   1.70974136]
 [ 0.78310437  0.50708484  0.36907508 {\ldots}  2.34064315  1.70974136
   1.4731532 ]
 [ 0.50708484  0.36907508  0.29021236 {\ldots}  1.70974136  1.4731532
   1.59144728]]

Shape X\_train:  (300, 99)
Shape X\_test:  (160, 99)

Training{\ldots}
Epoch 1/3
6/6 [==============================] - 17s 740ms/step - loss: 0.7312 - val\_loss:
0.2465
Epoch 2/3
6/6 [==============================] - 4s 744ms/step - loss: 0.3824 - val\_loss:
0.1302
Epoch 3/3
6/6 [==============================] - 2s 391ms/step - loss: 0.2226 - val\_loss:
0.1349

Predicting{\ldots}
5/5 [==============================] - 2s 96ms/step
    \end{Verbatim}

    \begin{center}
    \adjustimage{max size={0.9\linewidth}{0.9\paperheight}}{ICT607-lab-9-manual_files/ICT607-lab-9-manual_22_1.png}
    \end{center}
    { \hspace*{\fill} \\}
    
    \hypertarget{practice-task}{%
\subsection{Practice task}\label{practice-task}}

\begin{enumerate}
\def\labelenumi{\arabic{enumi}.}
\item
  Change the \texttt{train\_start}, \texttt{train\_end},
  \texttt{test\_start}, and \texttt{test\_end} parameters in the run
  function and observe how the generated graph changes in terms of
  anomaly detection. Try to increase the training data and decrease the
  test data or vice versa. Comment on how the changes affect the model's
  ability to detect anomalies.
\item
  Load the \texttt{cpu-full-a.csv} dataset and analyse the dataset to
  detect anomalies. Experiment with different hyperparameters to achieve
  better results. Finally, visualise the predictions and the actual data
  to understand how the model is performing.
\end{enumerate}


    % Add a bibliography block to the postdoc
    
    
    
\end{document}
